%%%%%%%%%%%%%%%%%%%%%%%%%%%%%%%%%%%%%%%%%
% University/School Laboratory Report
% LaTeX Template
% Version 3.0 (4/2/13)
%
% This template has been downloaded from:
% http://www.LaTeXTemplates.com
%
% Original author:
% Linux and Unix Users Group at Virginia Tech Wiki
% (https://vtluug.org/wiki/Example_LaTeX_chem_lab_report)
%
% License:
% CC BY-NC-SA 3.0 (http://creativecommons.org/licenses/by-nc-sa/3.0/)
%
%%%%%%%%%%%%%%%%%%%%%%%%%%%%%%%%%%%%%%%%%

%----------------------------------------------------------------------------------------
%	PACKAGES AND DOCUMENT CONFIGURATIONS
%----------------------------------------------------------------------------------------

\documentclass[a4paper]{article}

\usepackage{mhchem} % Package for chemical equation typesetting
\usepackage{siunitx} % Provides the \SI{}{} command for typesetting SI units

\usepackage{graphicx} % Required for the inclusion of images

\setlength\parindent{0pt} % Removes all indentation from paragraphs

\renewcommand{\labelenumi}{\alph{enumi}.} % Make numbering in the enumerate environment by letter rather than number (e.g. section 6)

%\usepackage{times} % Uncomment to use the Times New Roman font

%-------------------------------------------------------------------------------
%	DOCUMENT INFORMATION
%-------------------------------------------------------------------------------

\title{Project Definition: \\ Comparison of Matrix Factorization Based
Compression Methods Applied to Audio\\ Multimedia Communications HS 2013} %Title

\author{Philipp \textsc{Rohr} \\ 04-397-030} % Author name

\date{\today} % Date for the report

\begin{document}

\maketitle % Insert the title, author and date

% If you wish to include an abstract, uncomment the lines below
% \begin{abstract}
% Abstract text
% \end{abstract}

%-------------------------------------------------------------------------------
%	SECTION 1
%-------------------------------------------------------------------------------

\section{Motivation}

In the course Computational Intelligence Lab\footnote{http://cil.inf.ethz.ch/}
we used several matrix factorization approaches to compress images. These
approaches included Principal Component Analysis, Singular Value Decomposition,
Discrete Cosine Transformation, Wavelet Transformation and more. During the
course we developed quite some code to compress images and compared the
different approaches in terms of compression rate, error and speed. In my
project I would like to do a similar analysis for audio. I wonder whether the
results can be compared to image compression, meaning if they are similar in
error and compression rate or if they should not be applied to audio at all.

%-------------------------------------------------------------------------------
%	SECTION 2
%-------------------------------------------------------------------------------

\section{Approach} I plan to extend the matlab code base developed for the CIL
course to make it work with audio signals. I am going to implement at least the
DCT and the Wavelet Compression and compare them in terms of compression rate
and error. The audio samples that I am going to use are the ones that can be
found on the Opus codec website\footnote{http://www.opus-codec.org/examples/},
one for speech and one with different music styles.


%-------------------------------------------------------------------------------
%	SECTION 3
%-------------------------------------------------------------------------------

\section{Expected Outcome}

In CIL we've learned that DCT shall be better for continuous signals while
wavelet transformation can be better applied to compress localized signals. This
can be seen in image compression where the wavelet based JPEG2000 achieves
better results than its predecessor JPEG which is DCT based. Since audio has
more continuous characteristics I expect to achieve better results with DCT than
with the wavelet approach.

%-------------------------------------------------------------------------------
%	BIBLIOGRAPHY
%-------------------------------------------------------------------------------

%\bibliographystyle{unsrt}

%\bibliography{sample}

%-------------------------------------------------------------------------------


\end{document}