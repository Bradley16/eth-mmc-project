

\documentclass[a4paper]{scrartcl}

\usepackage{graphicx} % Required for the inclusion of images
\usepackage{xcolor}
\usepackage{epstopdf}
\usepackage{hyperref}


\title{Project Report: \\ Comparison of Matrix Factorization Based
Compression Methods Applied to Audio\\ Multimedia Communications HS 2013} %Title

\author{Philipp \textsc{Rohr} \\ 04-397-030} % Author name

\date{\today} % Date for the report

\begin{document}

\maketitle % Insert the title, author and date


\section{Introduction}
\subsection{Motivation}

In the course Computational Intelligence Lab\footnote{\url{http://cil.inf.ethz.ch/}}
we used several matrix factorization approaches to compress images. These
approaches included Principal Component Analysis, Singular Value Decomposition,
Discrete Cosine Transformation, Wavelet Transformation and more. During the
course we developed quite some code to compress images and compared the
different approaches in terms of compression rate, error and speed. In my
project I would like to do a similar analysis for audio. I wonder whether the
results can be compared to image compression, meaning if they are similar in
error and compression rate or if they should not be applied to audio at all.


\subsection{Approach} I plan to extend the Matlab code base developed for the CIL
course to make it work with audio signals. I am going to implement at least the
DCT and the Wavelet Compression and compare them in terms of compression rate
and error. The audio samples that I am going to use are the ones that can be
found on the Opus codec website\footnote{\url{http://www.opus-codec.org/examples/}},
one for speech and one with different music styles.


\subsection{Expected Outcome}

In CIL we've learned that DCT shall be better for continuous signals while
wavelet transformation can be better applied to compress localized signals. This
can be seen in image compression where the wavelet based
JPEG2000\footnote{\url{http://iphome.hhi.de/wiegand/assets/pdfs/DIC\_JPEG\_2000\_07.pdf}}
achieves higher compression with remaining quality than its predecessor JPEG
which is DCT based. Since audio has more continuous characteristics I expect to
achieve better results with DCT than with the wavelet approach.

\section{Theoretical Background} 

In this work I am using two different methods of basis transformations to get a
sparser representation of an audio signal. The first method is a Discrete Cosine
Transformation (DCT) and the the second is a Wavelet Transformation. Each method
takes an input signal $x$ and an orthogonal basis $U$ and computes the linear
transformation $$y = U \cdot x$$ The compression then truncates small or
insignificant values of $y$ yielding the compressed $\hat{y}$. The decompression
computes the inverse transformation $$\hat{x} = U^T \cdot \hat{y}$$ yielding the
reconstruction of the original $x$ minus some error.

The performance of a method is measured by its compression rate
$$ rate=\frac{number\_of\_bytes(\hat{y})}{number\_of\_bytes(x)}$$ and its Peak Signal
to Noise Ratio $$PSNR=20 \cdot log_{10}\left(\frac{max(x)}{\sqrt{MSE}}\right)$$
with $$ MSE = mean((x - \hat{x})^2)$$

\subsection{DCT compression} The DCT method partitions the input signal into
samples of a particular size (in my work 200ms) and transforms the samples into
its frequency domain using the DCT matrix $U$. In that frequency domain
frequencies with low amplitude are then cut away yielding only the few
significant frequencies.

\subsection{Wavelet Compression} The Wavelet compression also transforms the
input signal into the Wavlet space but does not sample it. The method
implemented in this work uses a
Haar\footnote{\url{http://de.wikipedia.org/wiki/Haar-Wavelet}} basis. To perform the
transformation the signal needs to be shaped into a square matrix with side
length of a power of 2.

\section{Implementation}
\label{sec:implementation}

The implementation can be found in folder \texttt{code}: 

\begin{itemize} 

\item \texttt{Benchmark.m} contains two benchmarks. One for DCT and one for
Wavelets. Each benchmark repeatedly calls \texttt{EvaluateCompression} with a
varying compression \texttt{factor}. The resulting compression rates and Peak
Signal to Noise Ratio (PSNR) are saved into a corresponding \texttt{.dat} file.

\item \texttt{EvaluateCompression.m} contains the infrastructure code that reads
audio signals from the \texttt{audio\_samples} folder, compresses the audio with
the given compression algorithm, decompresses it and calculates PSNR and
compression rates.

\item \texttt{DCTCompress.m and DCTDecompress.m} contain the
compression and decompression code for DCT which is explained in more detail in
\autoref{sec:dctCompression}.

\item \texttt{WaveletCompress.m and WaveletDecompress.m} contain the
compression and decompression code with Wavelet Transformation explained in more
detail in \autoref{sec:waveletCompression}.

\item \texttt{haarTrans.m} is a helper file that sets up an $NxN$
Haar transformation matrix.
\end{itemize}

\subsection{DCT Compression}
\label{sec:dctCompression}

\paragraph{Compression} receives input signal \texttt{W}, sample rate
\texttt{Fs} and a compression \texttt{factor}. It returns a data structure
\texttt{W\_comp} containing a sparse matrix and some auxiliary data. Following
steps are performed:

\begin{enumerate}

\item set $\texttt{sampleSize} = \frac{\texttt{Fs}}{5}$, means having a sample of
length 200 ms

\item reshape \texttt{W} into a matrix $R$ with size $\texttt{sampleSize} \times
N$

\item get the DCT basis matrix $U$ with \texttt{dctmtx}

\item transform $R$ to: $Y = U \cdot R$

\item compute a $threshold = mean(abs(Y)) \cdot \texttt{factor}$

\item set all elements $i$ of $Y$ where $Y_i < threshold$ to $0$

\item save \texttt{sparse(Y)} into return value \texttt{W\_comp.Y}


\end{enumerate}


\paragraph{Decompression} receives the compressed signal \texttt{W\_comp} and
returns the decompressed \texttt{W\_rec} performing following steps:

\begin{enumerate}

\item get the DCT basis matrix $U$ with \texttt{dctmtx}

\item compute the transformation $\texttt{W\_rec} = U^T \cdot
\texttt{W\_comp}.Y$

\item reshape \texttt{W\_rec} to its original dimensions and return it

\end{enumerate}

\subsection{Wavelet Compression}
\label{sec:waveletCompression}

\paragraph{Compression} receives input signal \texttt{W}, sample rate
\texttt{Fs} and a compression \texttt{factor}. It returns a data structure
\texttt{W\_comp} containing a sparse matrix and some auxiliary data. Following
steps are performed:

\begin{enumerate}

\item reshape \texttt{W} into a squared matrix $R$ having width and length
powers of 2

\item get the Wavelet basis matrix $U$ with \texttt{haarTrans}

\item transform $R$ to: $Y = U \cdot R$

\item compute a $threshold = mean(abs(Y)) \cdot \texttt{factor}$

\item set all elements $i$ of $Y$ where $Y_i < threshold$ to $0$

\item save \texttt{sparse(Y)} into return value \texttt{W\_comp.Y}


\end{enumerate}


\paragraph{Decompression} receives the compressed signal \texttt{W\_comp} and
returns the decompressed \texttt{W\_rec} performing following steps:

\begin{enumerate}

\item get the Wavelet basis matrix $U$ with \texttt{haarTrans}

\item compute the transformation $\texttt{W\_rec} = U^T \cdot
\texttt{W\_comp}.Y$

\item reshape \texttt{W\_rec} to its original dimensions and return it

\end{enumerate}

\newpage
\section{Experiments and Results}

The benchmark described in \autoref{sec:implementation} was applied to an audio
signal containing speech and one containing music, both obtained from the Opus
codec website. The benchmark computes PSNR and compression rate at a given
compression factor. Increasing the compression factor in the benchmarks also
increases the PSNR and compression rates. Figures \ref{fig:music_performance}
and \ref{fig:speech_performance} show the performance of the compression methods
applied to the different audio signals.

In compression method comparisons one often sees the comparison of PSNR and
bitrate. Since my wavelet compression algorithm cannot be applied to samples
rather it has to be applied to the whole audio signal to decompress it
at once, I decided to measure only the overall compression rate and not bitrates.

\begin{figure}[h!]
\centering
\begin{minipage}{0.5\textwidth}
\centering
\includegraphics[width=1\textwidth]{figures/music.eps}
\caption{performance on music}
\label{fig:music_performance}
\end{minipage}\hfill
\begin{minipage}{0.5\textwidth}
\centering
\includegraphics[width=1\textwidth]{figures/speech.eps}
\caption{performance on speech}
\label{fig:speech_performance}
\end{minipage}
\end{figure}
\section{Discussion} As expected in the introduction the DCT-based compression
results in better compression rates than the Wavelet-based approach. On the
other hand one can say that it achieves better quality at a fixed compression
rate. When listening to the decompressed speech files the ones having a PSNR
above 45 dB sound nearly transparent to my ears while I could hear more and more
artefacts with higher compression. For the decompressed music I had to go to
about 55 dB to not hear any more artefacts.

An other advantage of the DCT-based approach is its ability to decompress sample
by sample. This is due to keeping the sample order in the compressed data
structure and thus also keeping the time semantics. This means it could be
applied to a music player that decompresses only a few samples at a time which
then get to be played. The wavelet-based approach does not keep the order of the
samples, since the signal is reshaped to a square matrix and thus an audio
player would have to decompress the audio signal fully, before being able to
play it.

\paragraph{Room for Improvement} There are some things that could be improved in
some future work:

\begin{itemize}

\item My work does not look at any sort of psychoacoustic models. One could have
a look into improving the methods in that direction. This means finding a better
way to get rid of inaudible frequencies in both compression methods.

\item To actually compress the matrices I use the \texttt{sparse(x)} function of
Matlab. To achieve higher compression ratios it would make sense to replace
\texttt{sparse(x)} with some kind of entropy coding.

\item An other idea is to find a better way to apply wavlet based methods to
samples rather to the whole audio file itself.

\end{itemize}
\section{Summary and Conclusion}

In this work I compared a DCT with a Wavelet-based compression method that were
applied to audio signals. The basic code fragments were developed in the
Computational Intelligence Lab for images and had to be adapted to audio. My
work looks at the relationship between quality (PSNR) and compression rate of
the compressed signals. It shows that the DCT-based approach results in better
compression when having the same quality.

It was interesting to learn the basic principals of audio compression and to
hear the differences that appear when different methods get applied to the same
audio signal. It is always fun to not only see some numbers but to
actually hear (and thus feel) what those numbers mean.

\end{document}
